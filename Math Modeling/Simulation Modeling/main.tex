\documentclass[legalpaper,11pt,extrafontsizes,oneside,openany,x11names]{memoir}

\usepackage{geometry}
\geometry{portrait,left=1in,right=1in,top=1in,bottom=1in}

\usepackage{amsmath,booktabs}
\usepackage{graphicx,caption}
\usepackage{enumitem,multicol}
\usepackage{fancyhdr}
\usepackage{titlesec}
\usepackage{listings,xcolor}

\usepackage[
    pdftitle={Simulation Modeling},
    pdfauthor={Tamira Laki},
    colorlinks=true,
    urlcolor=primaryColor
]{hyperref}

\definecolor{codegray}{rgb}{0.9,0.9,0.9}

\lstset{
    backgroundcolor=\color{codegray},
    basicstyle=\ttfamily\small,
    breaklines=true,
    frame=single,
    captionpos=b,
    language=Matlab
}

\setlength{\itemsep}{0pt}
\setlength{\topsep}{0pt}
\setlength{\parsep}{0pt}
\setlength{\parskip}{0pt}

\titlespacing*{\section}{0pt}{1.5ex plus .2ex minus .2ex}{1ex plus .2ex}
\titlespacing*{\subsection}{0pt}{1.5ex plus .2ex minus .2ex}{0.5ex plus .2ex}
\titlespacing*{\subsubsection}{0pt}{1.5ex plus .2ex minus .2ex}{0.5ex plus .2ex}

\setlength{\columnsep}{20pt}

\fancypagestyle{plain}{
    \fancyhead[L]{\sffamily \textbf{LAKI, Tamira Marie A.}}
    \fancyhead[C]{\sffamily \textbf{Math Modeling (MAT305)}}
    \fancyhead[R]{\sffamily \textbf{BSM CS 3A-G2}}
    \fancyfoot[C]{\thepage}
}

\parindent0in
\pagestyle{plain}
\thispagestyle{plain}

\makeatletter
\def\maketitle{
    \begin{center}
        \LARGE \textbf{\@title} \\[0.5ex]
        \large \@date
    \end{center}
}
\makeatother

\title{Simulation Modeling}
\date{\today}

\begin{document}

\maketitle


\section*{Activity: Simulating Deterministic Behavior}

\begin{enumerate}
    \item Use Monte Carlo simulation to approximate the area under the curve \( y = \sqrt{x} \), over the interval \( \frac{1}{2} \leq x \leq \frac{3}{2} \). Give the actual area using the definite integral.

     \textbf{MATLAB Code:}
    \begin{lstlisting}
a = 0.5;
b = 1.5;
fmax = sqrt(b);
total_points = 100000;

x_rand = a + (b - a) * rand(1, N);     
y_rand = fmax * rand(1, N);            

under_curve = y_rand <= sqrt(x_rand);
approx_area = sum(under_curve) / total_points * (b - a) * fmax;

actual_area = (2/3) * ((b)^(3/2) - (a)^(3/2));

fprintf(actual_area);
    \end{lstlisting}

    \textbf{Output:}

    \begin{lstlisting}
0.98904
    \end{lstlisting}


\textbf{Solution:}
\begin{align*}
\text{A} 
&= \int_{1/2}^{3/2} \sqrt{x} \, dx = \left. \frac{2}{3} x^{3/2} \right|_{1/2}^{3/2} \\
&= \frac{2}{3} \left( \left( \frac{3}{2} \right)^{3/2} - \left( \frac{1}{2} \right)^{3/2} \right) \\
&= \frac{2}{3} (1.8371 - 0.3536) \\
&= \frac{2}{3} (1.4835) \approx 0.9890
\end{align*}

As shown in the MATLAB simulation and in the step-by-step solution, the actual area is approximately \boxed{0.9890}

    
    \item Use Monte Carlo simulation. Write an algorithm to calculate the part of the volume of an ellipsoid 
    \[
    \frac{x^2}{2} + \frac{y^2}{4} + \frac{z^2}{8} \leq 16
    \]
    that lies in the first octant, \( x > 0, y > 0, z > 0 \).

    \textbf{MATLAB Code:}
    \begin{lstlisting}
x_max = sqrt(32);
y_max = sqrt(64);
z_max = sqrt(128);
total_points = 100000;

x = x_max * rand(1, total_points);
y = y_max * rand(1, total_points);
z = z_max * rand(1, total_points);

inside = (x.^2)/2 + (y.^2)/4 + (z.^2)/8 <= 16;

box_volume = x_max * y_max * z_max;
volume = (sum(inside) / total_points) * box_volume;

fprintf(volume);
    \end{lstlisting}

    \textbf{Output:}
    \begin{lstlisting}
268.07808
    \end{lstlisting}

\textbf{Solution:}
    \begin{align*}
V 
&= \frac{1}{8} \cdot \frac{4}{3} \pi \cdot \sqrt{32} \cdot \sqrt{64} \cdot \sqrt{128} \\
&= \frac{1}{8} \cdot \frac{4}{3} \pi \cdot (2^{2.5}) (2^3) (2^{3.5}) \\
&= \frac{1}{8} \cdot \frac{4}{3} \pi \cdot 2^9 \\
&= \frac{1}{8} \cdot \frac{2048}{3} \pi \\
&= \frac{256}{3} \pi \\
&\approx 268.08
\end{align*}

The calculated part of the volume of the ellipsoid is approximately \boxed{268.08}, using MATLAB simulation and the step-by-step solution as the reference.

    
\end{enumerate}

\section*{Activity: Generating Random Numbers}

\begin{enumerate}
    \item Use middle square method to generate 10 random numbers using \( x = 1009 \). \\
\textbf{MATLAB Code:}
\begin{lstlisting}
x = 1009;
total_points = 10;

fprintf(' i     x_i   x_i^2\n');
fprintf('-------------------\n');

for i = 0:total_points
    x_sq_str = sprintf('%08d', x^2); 
    mid = floor((length(x_sq_str) - 4) / 2) + 1;
    middle_digits = x_sq_str(mid:mid+3); 
    
    fprintf('%2d  %5d   %s\n', i, x, middle_digits);
    x = str2double(middle_digits);
end
\end{lstlisting}

\textbf{Output:}
\begin{lstlisting}
 i     x_i   x_i^2
-------------------
 0   1009   1808
 1    180   0324
 2    324   1049
 3   1049   1004
 4   1004   0080
 5     80   0064
 6     64   0040
 7     40   0016
 8     16   0002
 9      2   0000
10      0   0000
\end{lstlisting}

\textbf{Solution:}
\begin{align*}
x_0 &= 1009 \\
x_0^2 &= 1018081 \Rightarrow 01018081 \Rightarrow x_1 = 0180 \\
x_1^2 &= 32400 \Rightarrow 00032400 \Rightarrow x_2 = 0324 \\
x_2^2 &= 104976 \Rightarrow 00104976 \Rightarrow x_3 = 1049 \\
x_3^2 &= 1100401 \Rightarrow 01100401 \Rightarrow x_4 = 1004 \\
x_4^2 &= 1008016 \Rightarrow 01008016 \Rightarrow x_5 = 0080 \\
x_5^2 &= 6400 \Rightarrow 00006400 \Rightarrow x_6 = 0064 \\
x_6^2 &= 4096 \Rightarrow 00004096 \Rightarrow x_7 = 0040 \\
x_7^2 &= 1600 \Rightarrow 00001600 \Rightarrow x_8 = 0016 \\
x_8^2 &= 256 \Rightarrow 00000256 \Rightarrow x_9 = 0002 \\
x_9^2 &= 4 \Rightarrow 00000004 \Rightarrow x_{10} = 0000 \\
\end{align*}

\newpage

    \item Use the linear congruence method to generate 20 random numbers using \( a = 5 \), \( b = 3 \), and \( c = 16 \). Was there cycling? If so, when did it occur?

    \textbf{Solution:}
\begin{align*}
x_0 & = 1 \\
x_1 & = (5 \cdot 1 + 3) \bmod 16 = 8 \\
x_2 & = (5 \cdot 8 + 3) \bmod 16 = 11 \\
x_3 & = (5 \cdot 11 + 3) \bmod 16 = 10 \\
x_4 & = (5 \cdot 10 + 3) \bmod 16 = 5 \\
x_5 & = (5 \cdot 5 + 3) \bmod 16 = 12 \\
x_6 & = (5 \cdot 12 + 3) \bmod 16 = 15 \\
x_7 & = (5 \cdot 15 + 3) \bmod 16 = 14 \\
x_8 & = (5 \cdot 14 + 3) \bmod 16 = 9 \\
x_9 & = (5 \cdot 9 + 3) \bmod 16 = 0 \\
x_{10} & = (5 \cdot 0 + 3) \bmod 16 = 3 \\
x_{11} & = (5 \cdot 3 + 3) \bmod 16 = 2 \\
x_{12} & = (5 \cdot 2 + 3) \bmod 16 = 13 \\
x_{13} & = (5 \cdot 13 + 3) \bmod 16 = 4 \\
x_{14} & = (5 \cdot 4 + 3) \bmod 16 = 7 \\
x_{15} & = (5 \cdot 7 + 3) \bmod 16 = 6 \\
x_{16} & = \boxed{(5 \cdot 6 + 3) \bmod 16 = 1}
\end{align*}

\textbf{Yes, and the cycle starts at $x_{16}$}, which repeats the initial value.


\end{enumerate}

\section*{Activity: Simulating Probabilistic Behavior}
\begin{enumerate}
    \item You arrive at the beach for a vacation and are dismayed to learn that PAGASA is predicting 50\% chance of rain every day. Using Monte Carlo simulation, predict the chance that it rains three consecutive days during your vacation.



\textbf{MATLAB Code:}
\begin{lstlisting}
total_points = 100000;
count = 0;

for i = 1: total_points
    day1 = rand < 0.5;
    day2 = rand < 0.5;
    day3 = rand < 0.5;

    if day1 && day2 && day3
        count = count + 1;
    end
end

prob = count / total_points;
fprintf(total_points, prob, prob * 100);
\end{lstlisting}

\textbf{Output:}

\begin{lstlisting}
0.1249 (12.49%)
\end{lstlisting}

\textbf{Solution:}
\begin{align*}
P(\text{rain each day}) 
&= P(\text{rain on day 1}) \cdot P(\text{rain on day 2}) \cdot P(\text{rain on day 3}) \\
&= 0.5 \cdot 0.5 \cdot 0.5 = \boxed{0.125}
\end{align*}

As displayed in the MATLAB simulation and step-by-step solution for the probability, there is approximately \boxed{$12.50\%$} chance that it rains for 3 consecutive days during my vacation.
\end{enumerate}

\end{document}
